The assignment requires dealing with a stream of sensor readings. As we do not have
such data streams, we need to provide dummy sources of sensor readings. This will 
be done by implementing server-side dummy sensors that communicate with client-side
(the browser). 

To this end {\bf the sensors will be registered in server-side database} and
the server {\bf will continuously provide their readings} to the clients.

\begin{figure}[h]
\begin{lstlisting}[language=json,firstnumber=1]
    {
        "msgkind": "reading",
        "sensor": <put sensor name here>,
        "reading": <the numeric (can be 0-1 boolean) value sent by the sensor>,
        "unit": <the unit of measured quantity, e.g. kph, decibels, or none>,
        "location": <either a string with location name or numeric coordinates>
    }
\end{lstlisting}
\caption{Abstract sensor reading: JSON definition}
\label{fig:abstract_sensor}
\vspace{4mm}
\begin{lstlisting}[language=json,firstnumber=1]
    {
        "msgkind": "reading",
        "sensor": "noise sensor",
        "reading": 50,
        "unit": "db",
        "location": "Room C124, Henry Grattan Building, Glasnevin Campus"
    }
\end{lstlisting}
\caption{Noise sensor reading: JSON definition}
\label{fig:noisereading}
\vspace{4mm}
\begin{lstlisting}[language=json,firstnumber=1]
    {
        "msgkind": "reading",
        "sensor": "parking sensor",
        "reading": 1,
        "unit": null,
        "location": "Car Park 1, spot 113" 
    }
\end{lstlisting}
\caption{Parking spot sensor reading (busy): JSON definition}
\label{fig:parkingspotreading}
\end{figure}

\cref{fig:abstract_sensor} shows the template for abstract sensor readings
while \cref{fig:noisereading} and \cref{fig:parkingspotreading} show how this
can be specialized to two concrete cases.

\section{Server-side events}

We need to be able to:
\begin{itemize}
    \item Register a new sensor in the database. This registration should provide all the details
    shown in JSON files above, plus valid reading ranges. 
    \item Generate (every few seconds?) readings, probed from the reading ranges.
    \item Send the readings to the clients, using a mechanism yet to be identified.
\end{itemize}

Concerning the final point above, this: \url{https://stackoverflow.com/questions/5195452/websockets-vs-server-sent-events-eventsource}
suggests that either WebSockets or Server-Side Events can be used.